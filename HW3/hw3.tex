\documentclass[12pt,letterpaper]{article}
\usepackage{fullpage}
\usepackage[top=2cm, bottom=4.5cm, left=2.5cm, right=2.5cm]{geometry}
\usepackage{amsmath,amsthm,amsfonts,amssymb,amscd}
\usepackage{lastpage}
\usepackage{enumerate}
\usepackage{fancyhdr}
\usepackage{mathrsfs}
\usepackage{xcolor}
\usepackage{graphicx}
\usepackage{listings}
\usepackage{hyperref}
\usepackage{float}
\usepackage{enumitem}

\hypersetup{%
  colorlinks=true,
  linkcolor=blue,
  linkbordercolor={0 0 1}
}
 
\renewcommand\lstlistingname{Algorithm}
\renewcommand\lstlistlistingname{Algorithms}
\def\lstlistingautorefname{Alg.}

\lstdefinestyle{Python}{
    language        = Python,
    frame           = lines, 
    basicstyle      = \footnotesize,
    keywordstyle    = \color{blue},
    stringstyle     = \color{green},
    commentstyle    = \color{red}\ttfamily
}

\setlength{\parindent}{0.0in}
\setlength{\parskip}{0.05in}

% Edit these as appropriate
\newcommand\course{CSE 3666}
\newcommand\hwnumber{2}                  % <-- homework number
\newcommand\NetIDa{alb20017}           % <-- NetID of person #1

\pagestyle{fancyplain}
\headheight 35pt
\lhead{\NetIDa}
\chead{\textbf{\Large Homework \hwnumber}}
\rhead{\course \\ \today}
\lfoot{}
\cfoot{}
\rfoot{\small\thepage}
\headsep 1.5em

\begin{document}

\section*{Question 1}

    \lstset{label={lst:alg1}}
     \begin{lstlisting}[style = Python]
foo:
	add	s1, x0, x0	# count = s1 = 0
	add	s2, x0, x0	# i = s2 = 0
	add	s3, a0, x0	# d = s3 = a0
	add	s4, a1, x0	# n = s4 = a1
	
forloop:
	beq	s2, s4, done	# once looping done go to done
	
	slli	a0, s2, 2	# a0 = &d[i]
	sub	a1, s4, s2	# a1 = n-i
	add	s5, ra, x0	# save foo ra in s5 before bar call
	jal	ra, bar		# call bar after setting a0 and a1 first
	add	ra, s5, x0	# set ra back
	blt	x0, s6, incriment	# assume bar saves t to s6
	addi	s2, s2, 1	# i+=1
	beq	x0, x0, forloop

icriment:
	addi	s1, s1, 1	# count += 1
	addi	s2, s2, 1	# i+=1
	beq	x0, x0, forloop
	
done:
	jalr x0, ra, 0

 \end{lstlisting}

\newpage

\section*{Question 2}



    \lstset{label={lst:alg1}}
     \begin{lstlisting}[style = Python]
msort:
    addi 	sp, sp, -16   # Make space on stack for 4 registers (4 * 4 bytes each)

    sw 		ra, 0(sp)       # Save return address
    sw 		s0, 4(sp)       # Save s0 register
    sw 		s1, 8(sp)       # Save s1 register
    sw 		s2, 12(sp)      # Save s2 register


    lw 		s0, 0(a0)       # s0 = d (array to be sorted)
    lw		s1, 4(a0)       # s1 = n (size of the array)


    addi 	t1, x0, 1
    bge 	s1, done

    srli 	s2, s1, 1      # s2 = n / 2 (n1)
    add 	s2, s0, s2      # s2 = &d[n1]

    jal 	ra, msort       # msort(d, n1)
    add 	a0, s2, x0           # Set new arguments for the second recursive call
    sub 	a1, s1, s2      # n - n1
    jal 	ra, msort       # msort(&d[n1], n - n1)


    lw 		a0, 0(sp)        # Restore d
    lw 		a1, 8(sp)        # Restore n1
    lw 		a2, 12(sp)       # Restore &d[n1]
    jal 	ra, merge       # merge(c, d, n1, &d[n1], n - n1)


    lw 		a0, 0(sp)        # Restore d
    lw 		a1, 4(sp)        # Restore c
    lw 		a2, 8(sp)        # Restore n
    jal 	ra, copy        # copy(d, c, n)

    exit:
    lw 		ra, 0(sp)        # Restore return address
    lw 		s0, 4(sp)        # Restore s0 register
    lw 		s1, 8(sp)        # Restore s1 register
    lw 		s2, 12(sp)       # Restore s2 register

    addi sp, sp, 16

    \end{lstlisting}

\newpage
\section*{Question 3}

\begin{enumerate}
\item
BGE x10, x20, I100
\\It is an SB-type instruction
\\imm: 000010110100
\\opcode: 1100011
\\funct3: 101
\\rs1: 01010
\\rs2: 10100
\\machine code in bits: 00010111010001010101010001100011		          
\\machine code in hex: 17455463

\item
BEQ x10, x0, I1
\\It is an SB-type instruction
\\imm: 111111101100
\\opcode: 1100011
\\funct3: 000
\\rs1: 01010
\\rs2: 00000
\\machine code in bits: 11111100000001010000110011100011
\\machine code in hex: FC050CE3

\item
JAL x0, I100
\\It is an UJ-type instruction
\\imm: 11111111111110110000
\\opcode: 1101111
\\rd: 00000
\\machine code in bits: 11110110000111111111000001101111
\\machine code in hex: F61FF06F

\end{enumerate}
\newpage
\section*{Question 4}
\begin{enumerate}
\item
BEQ x20, x21, -600
\\It is an SB-type instruction
\\imm: 111011010100
\\opcode: 1100011
\\funct3: 000
\\rs1: 10100
\\rs2: 10101
\\machine code in bits: 1101101 10101 10100 000 01001 1100011
\\machine code in hex: DB5A04E3
\\ offset = 111011010100(0) = -600
\\ 0x0400366C - 600 = 0x04003414

\item
JAL x9, 
\\It is an UJ-type instruction
\\imm: 11111111111111010100
\\opcode: 1100011
\\rd: 01001
\\machine code in bits: 1 1111010100 1 11111111 00001 1101111
\\machine code in hex: FA9FF0EF
\\ offset = 11111111111111010100(0) = -88
\\ 0x04208888 - 88 = 0x04208830
\end{enumerate}


\end{document}