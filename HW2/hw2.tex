\documentclass[12pt,letterpaper]{article}
\usepackage{fullpage}
\usepackage[top=2cm, bottom=4.5cm, left=2.5cm, right=2.5cm]{geometry}
\usepackage{amsmath,amsthm,amsfonts,amssymb,amscd}
\usepackage{lastpage}
\usepackage{enumerate}
\usepackage{fancyhdr}
\usepackage{mathrsfs}
\usepackage{xcolor}
\usepackage{graphicx}
\usepackage{listings}
\usepackage{hyperref}
\usepackage{float}
\usepackage{enumitem}

\hypersetup{%
  colorlinks=true,
  linkcolor=blue,
  linkbordercolor={0 0 1}
}
 
\renewcommand\lstlistingname{Algorithm}
\renewcommand\lstlistlistingname{Algorithms}
\def\lstlistingautorefname{Alg.}

\lstdefinestyle{Python}{
    language        = Python,
    frame           = lines, 
    basicstyle      = \footnotesize,
    keywordstyle    = \color{blue},
    stringstyle     = \color{green},
    commentstyle    = \color{red}\ttfamily
}

\setlength{\parindent}{0.0in}
\setlength{\parskip}{0.05in}

% Edit these as appropriate
\newcommand\course{CSE 3666}
\newcommand\hwnumber{2}                  % <-- homework number
\newcommand\NetIDa{alb20017}           % <-- NetID of person #1

\pagestyle{fancyplain}
\headheight 35pt
\lhead{\NetIDa}
\chead{\textbf{\Large Homework \hwnumber}}
\rhead{\course \\ \today}
\lfoot{}
\cfoot{}
\rfoot{\small\thepage}
\headsep 1.5em

\begin{document}

\section*{Question 1}

\begin{enumerate}[label=\alph*]
  \item
    \lstset{label={lst:alg1}}
     \begin{lstlisting}[style = Python]
# RISC-V code
	addi 	s4, x0, 100
	addi 	s1, x0, 0

loop:
	slli 	t0, s1, 2 	# t0 = i * 4
	add 	t2, t0, s2 	# compute addr of A[i]
	lw 	t1, 0(t2)
	addi	t1, t1, 4	# add 4 to A[i]
	add 	t3, t0, s3 	# compute addr of B[i]
	sw 	t1, 0(t3)
	addi 	s1, s1, 1
	
test: 	bne s1, s4, loop 	# 8 instructions in the loop
    \end{lstlisting}

The amount of instructions that will run is 802. There are 8 instructions per loop, and 100 loops, plus the 2 initializing functions, resulting in 802 instructions total. The change that needed to be made was to add 4 to A[i] before loading it into B[i].

\newpage
  \item
    \lstset{label={lst:alg1}}
     \begin{lstlisting}[style = Python]
# RISC-V code
	addi 	s4, x0, 100
	addi 	s1, x0, 0

loop:
	# B[i] = A[i] + 4;
	slli 	t0, s1, 2 	# t0 = i * 4
	add 	t2, t0, s2 	# compute addr of A[i]
	lw 	t1, 0(t2)
	addi	t1, t1, 4	# add 4 to A[i]
	add 	t3, t0, s3	# compute addr of B[i]
	sw 	t1, 0(t3)
	
	# B[i+1] = A[i+1] + 4;
	lw 	t1, 4(t2)	# A[i+1]
	addi	t1, t1, 4	# add 4 to A[i+1]
	sw 	t1, 4(t3)	# B[i+1] = A[i+1] + 4;

	# B[i+2] = A[i+2] + 4;
	lw 	t1, 8(t2)	# A[i+2]
	addi	t1, t1, 4	# add 4 to A[i+2]
	sw 	t1, 8(t3)	# B[i+2] = A[i+2] + 4;
	
	# B[i+3] = A[i+3] + 4;
	lw 	t1, 12(t2)	# A[i+3]
	addi	t1, t1, 4	# add 4 to A[i+3]
	sw 	t1, 12(t3)	# B[i+3] = A[i+3] + 4;
	
	addi 	s1, s1, 4	# incriment by 4 each iteration
	
test: 	bne s1, s4, loop 	# 17 instructions in the loop
    \end{lstlisting}

The amount of instructions that will run is 427. There are 17 instructions per loop, and 25 loops, plus the 2 initializing functions, resulting in 427 instructions total. The code simply adjusts the offsets by 4 for each increase in index for A and B this works because their addresses are calculated at the beginning of each iteration, so the next element in them 4 bytes later. 

\end{enumerate}

\newpage

\section*{Question 2}



    \lstset{label={lst:alg1}}
     \begin{lstlisting}[style = Python]
	# init iloop iterators
	addi 	s0, x0, 0		# s0=i=0
	addi 	s1, x0, 16		# s1=16
iloop:
	# init jloop iterators
	addi 	s2, x0, 0		# s2=j=0
	addi 	s3, x0, 8		# s3=8
jloop:
	slli	t0, s0, 8		# t0 = 256 * i
	add 	t0, t0, s2		# t0 = 256 * i + j
	
	# calculate offset
	slli	t1, s0, 5		# t1 = i * 8 * 4 = i offset
	slli	t2, s2, 2		# t2 = j * 4
	add	t1, t1, t2		# t1 = total offset = i offset + j offset
	add	t1, t1, s9		# t1 = address of T[i][j]
	sw	t0, 0(t1)		# T[i][j] = 256 * i + j;

	addi	s2, s2 1		# incriment j
	bne 	s2, s3 iloop		# jloop if j!=8
	
	addi	s0, s0 1		# incriment i
	bne 	s0, s1 iloop		# iloop if i!=16

    \end{lstlisting}
Each loop's iterators are initialized before the loop is executed. At the end of the loop it incriments the iterator and checks if it is the last one, if so it will not start another loop. I calculated the location T[i][j] with the formula: (T address) + ( i * 8 * 4) + (j * 4). This works because i * 8 will calculate how many rows in, and multiply by 4 to account for the size of each element being 4 bytes, then add j to the row, again multiplying by 4 to account for the size, this will get the offset so all that needs to be done is to add it to the address of T.

\newpage
\section*{Question 3}

    \lstset{label={lst:alg1}}
     \begin{lstlisting}[style = Python]
# Addition of decimal strings

# strings are stored in global data section 
        .data   
dst:    .space  128
str1:   .space  128
str2:   .space  128

# instructions are in text section
        .text
main: 
        # load adresses of strings into s1, s2, and s3
        # s3 is dst, where we store the result 

        lui     s3, 0x10010 
        addi    s1, s3, 128
        addi    s2, s1, 128

        # read the first number as a string
        addi    a0, s1, 0
        addi    a1, x0, 100
        addi    a7, x0, 8
        ecall

        # read the second number as a string
        addi    a0, s2, 0
        addi    a1, x0, 100
        addi    a7, x0, 8
        ecall

        # useful constants
        addi    a4, x0, 0
        addi    a5, x0, 10

        #TODO
        # write a loop to find out the number of decimal digits in str1
        # the loop searches for the first character that is less than '0' 
        add	t0, x0, s1
count_digits:
	lb	t1, 0(t0)		# load current byte
	addi	t0, t0, 1
	bgt	t1, a4,	count_digits
	addi	t0, t0, -2
	sub	t0, t0, s1		# t0 = digits
	
	
	
	add	s4, t0, s3	# s4 = s3 address
	sb	a4, 0(s4)
	add	s1, t0, s1
	add	s2, t0, s2
	add	t5, x0, x0	# t5 = iterator
	add	s9, x0, x0	# x9 = carry
	addi	s1, s1, -1
	addi	s2, s2, -1
	addi	s4, s4, -1
addloop:
	beq	t5, t0, print
	# t1 = digit from s1
        lb	t1, 0(s1)
        addi	t1, t1, -48
        # t2 = digit from s2
        lb	t2, 0(s2)
	addi	t2, t2, -48
        # t3 = sum
        add	t3, t1, t2
        add	t3, t3, s9
        
	bge	t3, a5, carry
	addi	t3, t3, 48
	sb	t3, 0(s4)
	addi	s9, x0, 0
	beq	x0, x0, incriment

carry:
        addi	t3, t3, 38
        sb	t3, 0(s4)
	addi	s9, x0, 1

incriment:
	addi	s1, s1, -1
	addi	s2, s2, -1
	addi	s4, s4, -1
	addi	t5, t5, 1
	beq	x0, x0, addloop

print:
        addi    a0, s3, 0
        addi    a7, x0, 4
        ecall

        # exit
        addi    a7, x0, 10
        ecall


    \end{lstlisting}
The addition is done by starting at the last byte of each adress s1 and s2, found by counting the amount of digits in each. Then iterate backwards to the start, converting each byte to an integer and adding them then saving them back as a string character, keeping track of the carry for the next iteration. 


\section*{Question 4}

\begin{enumerate}
\item
Instruction: or s1, s2, s3
\\Find out register numbers: or x9, x18, x19
\\It is an R-type instruction
\\funct7: 0000000
\\opcode: 0110011
\\rd: 01001
\\funct3: 110
\\rs1: 10010
\\rs2: 10011
\\machine code in bits: 00000001001110010110010010110011
\\machine code in hex: 013964b3

\item
Instruction: slli t1, t2, 16
\\Find out register numbers: slli x6, x7, 16
\\It is an I-type instruction
\\Immediate: 000000010000
\\opcode: 0010011
\\rd: 00110
\\funct3: 001
\\rs1: 00111
\\machine code in bits: 00000001000000111001001100010011
\\machine code in hex: 01039313

\item
Instruction: xori x1, x1, -1
\\Find out register numbers: xori x1, x1, -1
\\It is an I-type instruction
\\Immediate: 111111111111
\\opcode: 0010011
\\rd: 00001
\\funct3: 100
\\rs1: 00001
\\machine code in bits: 11111111111100001100000010010011
\\machine code in hex: fff0c093

\item
Instruction: lw x2, -100(x3)
\\Find out register numbers: lw x2, -100(x3)
\\It is an I-type instruction
\\Immediate: 111110011100
\\opcode: 0000011
\\rd: 00010
\\funct3: 010
\\rs1: 00011
\\machine code in bits: 11111001110000011010000100000011
\\machine code in hex: f9cla103

\end{enumerate}

\section*{Question 5}
\begin{enumerate}
\item
Instruction: sw x10,-16(x25)
\\Find out register numbers: sw x10,-16(x25)
\\It is an S-type instruction
\\Immediate: 111111110000
\\opcode: 0100011
\\funct3: 010
\\rs1: 11001
\\rs2: 01010
\\machine code in bits: 11111110101011001010100000100011
\\machine code in hex: feaca823

\item
Instruction: addi x14,x4,64
\\Find out register numbers: addi x14,x4,64
\\It is an I-type instruction
\\Immediate: 000001000000
\\opcode: 0010011
\\rd: 01110
\\funct3: 000
\\rs1: 00100
\\machine code in bits: 00000100000000100000011100010011
\\machine code in hex: 04020713

\item
Instruction: and x23,x10,x5
\\Find out register numbers: and x23,x10,x5
\\It is an I-type instruction
\\funct7: 0000000
\\opcode: 0110011
\\rd: 10111
\\funct3: 111
\\rs1: 01010
\\rs2: 00101
\\machine code in bits: 00000000010101010111101110110011
\\machine code in hex: 00557bb3

\item
Instruction: srai x30,x31,20
\\Find out register numbers: srai x30,x31,20
\\It is an I-type instruction
\\Immediate: 010000010100
\\opcode: 0010011
\\rd: 11110
\\funct3: 101
\\rs1: 11111
\\machine code in bits: 01000001010011111101111100010011
\\machine code in hex: 414fdf13
\end{enumerate}


\end{document}