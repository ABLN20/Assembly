\documentclass[12pt,letterpaper]{article}
\usepackage{fullpage}
\usepackage[top=2cm, bottom=4.5cm, left=2.5cm, right=2.5cm]{geometry}
\usepackage{amsmath,amsthm,amsfonts,amssymb,amscd}
\usepackage{lastpage}
\usepackage{enumerate}
\usepackage{fancyhdr}
\usepackage{mathrsfs}
\usepackage{xcolor}
\usepackage{graphicx}
\usepackage{listings}
\usepackage{hyperref}
\usepackage{float}
\usepackage{enumitem}
\usepackage{booktabs}
\usepackage{tabularray}

\hypersetup{%
  colorlinks=true,
  linkcolor=blue,
  linkbordercolor={0 0 1}
}
 
\renewcommand\lstlistingname{Algorithm}
\renewcommand\lstlistlistingname{Algorithms}
\def\lstlistingautorefname{Alg.}

\lstdefinestyle{Python}{
    language        = Python,
    frame           = lines, 
    basicstyle      = \footnotesize,
    keywordstyle    = \color{blue},
    stringstyle     = \color{green},
    commentstyle    = \color{red}\ttfamily
}

\setlength{\parindent}{0.0in}
\setlength{\parskip}{0.05in}

% Edit these as appropriate
\newcommand\course{CSE 3666}
\newcommand\hwnumber{9}                  % <-- homework number
\newcommand\NetIDa{alb20017}           % <-- NetID of person #1

\pagestyle{fancyplain}
\headheight 35pt
\lhead{\NetIDa}
\chead{\textbf{\Large Lab \hwnumber}}
\rhead{\course \\ \today}
\lfoot{}
\cfoot{}
\rfoot{\small\thepage}
\headsep 1.5em

\begin{document}
\section*{Question 1}
Block offset: 2 bits (cache block size is 4 ), cach index: 3 bits (8 blocks), tag: 32 bits
\newline 
Cache index for the first two is 0 because both are in block 0. Cache tag wont change until it has gone through all blocks, block offset will incriment 4 times per block
\newline 
The hit miss outcome was .75 hit rate. It would miss once then hit three times, then repeat. 
% \usepackage{tabularray}
\begin{table}
\centering
\begin{tblr}{
}
Address    & Cache index & tag        & Block offset & Hit/Miss \\
0x10010000 & 0           & 0x00200200 & 0            & Miss     \\
0x10010004 & 0           & 0x00200200 & 1            & Hit      \\
0x10010008 & 0           & 0x00200200 & 2            & Hit      \\
0x1001000c & 0           & 0x00200200 & 3            & Hit      \\
0x10010010 & 1           & 0x00200200 & 0            & Miss     \\
0x10010014 & 1           & 0x00200200 & 1            & Hit      
\end{tblr}
\end{table}
\section*{Question 2}
I did not predict all the hit rates correctly, I did not assume for them to just jump at the 64 blocks, I thought it would grow in a linear way. If the size of warray is doubled, the 64 block size row will have different data, it will have a 75 hit rate instead,
\begin{table}
\centering
\begin{tabular}{llll}
Number of blocks & Cache size(bytes) & Hit rate(\%) & Miss count  \\
8                & 128               & 75           & 3072        \\
16               & 256               & 75           & 3072        \\
32               & 512               & 75           & 3072        \\
64               & 1024              & 99           & 64          \\
128              & 2048              & 99           & 64         
\end{tabular}
\end{table}

\end{document}